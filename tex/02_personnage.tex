
\section{Personnages}
\lettrine{L}{e} c\oe{}ur des jeux de rôle réside dans la possibilité de créer, améliorer et faire évoluer son propre personnage. Voilà comment ça fonctionne dans {\jedifont Starwars reloaded}. 

\subsubsection{Race}
Vous pouvez choisir pour votre personnage n’importe quelle race disponible dans l'univers de Starwars. Nos Univers Sauvages contiennent des règles complètes et autres informations de background pour toute nouvelle race apparaissant dans ces mondes. Les personnages humains commencent avec un Atout gratuit de leur choix.

\begin{quotebox}
	As you approach this template you get a sense that the blood and tears of many generations went into its making. A warm feeling welcomes you as you type your first words.
\end{quotebox}

\newpage % Acts as columbreak because of twocolumn option; for pagebreak use \clearpage

% For more columns, you can say \begin{dndtable}[your options here}.
% For instance, if you wanted three columns, you could say
% \begin{dndtable}{XXX}. The usual host of tabular parameters are
% aailable as well.
\header{Nice table}
\begin{dndtable}
   	\textbf{Table head}  & \textbf{Table head} \\
   	Some value  & Some value \\
   	Some value  & Some value \\
   	Some value  & Some value
\end{dndtable}

\begin{paperbox}{Do the Players need direction?}
	\lipsum[1]
\end{paperbox}

% You can optionally not include the background by saying
% begin{monsterboxnobg}
\begin{monsterbox}{Monster Foo}
	\textit{Small metasyntatic variable (golbinoid), neutral evil}\\
	\hline
	\basics[%
	armorclass = 12,
	hitpoints  = 16 (3d8 + 3),
	speed      = 50 ft
	]
	\hline
	\stats[
    STR = \stat{12}, % This stat command will autocomplete the modifier for you
    DEX = \stat{7}
	]
	\hline
	\details[%
	% If you want to use commas in these sections, enclose the
	% description in braces.
	% I'm so sorry.
	languages = {Common Lisp, Erlang},
	]
	\hline \\[1mm]
	\begin{monsteraction}[Monster-super-powers]
		This Monster has some serious superpowers!
	\end{monsteraction}
	\monstersection{Actions}
	\begin{monsteraction}[Generate text]
		This one can generate tremendous amounts of text! Though only when it wants to.
	\end{monsteraction}

	\begin{monsteraction}[More actions]
    See, here he goes again! Yet more text.
	\end{monsteraction}
\end{monsterbox}
