\section{La Force}
\label{sec:force}

Abordons le chapitre de La Force. Si vous souhaiter jouer un Jedi ou un Sith c’est ici que cela se passe. Mais d’abord voyez avec votre MJ s’il est d’accord.

Avant de rentrer dans le vif du sujet, il est nécessaire de donner quelques avertissements. La création d’un Jedi est volontairement difficile et demande une bonne dose d’expérience. En premier lieu parce que les Jedi et les Sith ne consitituent une population très répendu. En second lieu, parce qu’une fois que vous êtes Jedi ou Sith, il n’y a plus grand chose qui va pouvoir vous résister en dehors d’autres Jedi et Sith. Du coup le fun en prend un coup, les variantes d’adversaires chuttant drastiquement.

Ceci dit avoir un Jedi dans l’équipe peu aussi ouvrir des orizons nouveau au MJ.

\subsection{Créer un Jedi/Sith}
La première chose pour créer un héro sensible à la force est de prendre l’atout \emph{Arcane (Force)} qui représente la faculté de votre héro à percevoir La Force. Cet atout ne peut être pris qu'à la création du personnage.

Ensuite vous devez investir des points dans la compétence \emph{Maîtrise de la Force} qui représente votre niveau de sensibilité à La Force et voter maîtrise de cette dernière.

Les héros utilisant la Force le font en consommant des Points de Pouvoir. Vous recevez \textbf{10} points de pouvoir à la création de votre personnage. L’utilisation de certains pouvoir (actif) nécessite de dépenser ces points de pouvoir. Un héros gagne un point de pouvoir par heure.

Les utilisateurs de la Force débutent avec \textbf{2}. Un utilisateur de la Force peut apprendre un nouveau pouvoir en choisissant l’Atout Nouveau pouvoir. Dès qu’il choisit cet Atout lors d’une Progression, il peut immédiatement utiliser le nouveau pouvoir sélectionné. Attention, certain pouvoir nécessitent d'avoir l'atout \emph{Padawan} ou \emph{Apprenti Sith}, d'autre d'être un Jedi ou un Sith.

\subsection{Les pouvoirs}

\subsubsection{Sens de Force (Lumineux)}
Appartenant à la catégorie des pouvoirs élémentaires, le pouvoir appelé sens de Force permet à un adepte de ressentir les émotions et les sentiments d’autres êtres vivants à travers la Force. Mais ce pouvoir ne se limite pas à cette simple perception ; cette technique est tout aussi pratique pour décrypter un avenir toujours changeant. Le sens de Force est aussi un moyen de ressentir les grandes secousses pouvant se produire dans la Force, comme la destruction d’une planète, entraînant la mort de millions d’individus. Enfin, cette technique peut également déceler la présence du Côté Obscur. 

\subsubsection{Vision de Force (Lumineux)}
La vision de Force est une capacité assez simple qui peut être assimilée d’une façon plus générale au sens de Force. Toutefois, il faut faire la part des choses entre la vision de Force et le sens de Force. En effet, le sens de Force ne permet à son utilisateur que d’obtenir une perception assez générale, de son environnement, alors que la vision de Force permet d’augmenter l’acuité de ce que voit son utilisateur, peu importe l’environnement dans lequel il se trouve. De même, en se focalisant sur une sensation précise, l’utilisateur de la vision de Force peut même voir à travers les murs. 