\subsection{Atouts}

Voici le liste des Atouts disponibles dans \swfe.

\subsubsection{Background}
\begin{description}[align=left]
    \item [Ambidextre]
    	\emph{[Novice, Agi d8+]}\\
        Votre héros utilise ses deux mains avec la même facilité.

    \item [Arcane (Force)]
    	\emph{[Novice, spécial, \^Ame d8+]}\\
        L'atout indispensable si votre héro doit utiliser la Force.

    \item [Brave]
    	\emph{[Novice, spécial]}\\
        Ceux qui possèdent cet Atout ont appris à maîtriser leurs peurs.

    \item [Chanceux]
    	\emph{[Novice]}\\
        Votre héros semble être béni par le destin.

    \item [Très Chanceux]
    	\emph{[Novice, Chanceux]}\\
        Votre héros semble être béni par le destin. Deux fois.

    \item [Costaud]
    	\emph{[Novice, Force et Vigueur d6+]}\\
        Votre héros est très corpulent ou tout simplement très athlétique.

    \item [Don des langues]
    	\emph{[Novice, Intellect d6+]}\\
        Le personnage a un don pour les langues.

    \item [Enragé]
    	\emph{[Novice]}\\
        Dès qu’il est blessé (ou même Secoué par une attaque physique) votre héros doit réussir un jet d’Intellect sans quoi il devient Enragé. Un personnage enragé attaque sans aucune retenue.

    \item [Guérison rapide]
    	\emph{[Novice, Vigueur d8+]}\\
        Votre héros récupère vite de ses blessures.

    \item [Noble]
    	\emph{[Novice]}\\
        Les personnages de haute naissance sont avantagés par la vie mais ont aussi plus de responsabilités.

    \item [Résistance à la Force]
    	\emph{[Novice, \^Ame d8+]}\\
        Votre héro, même s'il ne sait pas l'utiliser, est capable de résister à la Force quand elle est employé contre lui.

    \item [Grande résistance à la Force]
    	\emph{[Novice, Résistance à la Force]}\\
        Comme précédent mais avec un bonus de résistance double.

    \item [Riche]
    	\emph{[Novice]}\\
        Que votre héros soit né avec une cuillère en argent dans la bouche ou bien qu’il ait bien réussi en affaires, une chose est sûre : il est beaucoup plus fortuné que la plupart des gens.

    \item [Très riche]
    	\emph{[Novice, Riche ou Noble]}\\
        Votre héros est riche comme Crésus.

    \item [Séduisant]
    	\emph{[Novice, Vigueur d6+]}\\
        Votre héros a beaucoup de charme ou est très beau.

    \item [Très séduisant]
    	\emph{[Novice, Séduisant]}\\
        Votre héros est d’une beauté à couper le souffle.

    \item [Véloce]
    	\emph{[Novice, Agilité d6+]}\\
        Le héros se déplace très rapidement.

    \item [Vif]
    	\emph{[Novice, Agilité d8+]}\\
        Votre héros est né avec des réflexes presque surhumains.

    \item [Vigilant]
    	\emph{[Novice]}\\
        Peu de choses échappent à votre héros. Il est vigilant et très observateur.
\end{description}

\begin{paperbox}{Arcane (Force)}
    Remplace l'attout Arcanes. C'est l'atout qui représente la Force, l\^Ame étant assimilé au niveau de maîtrise de la Force.
\end{paperbox}

\begin{paperbox}{Résistance à la Force}
    Comme sont niveau supérieur Grande Résistance à la Force, cet atout remplace la résistance aux Arcanes. il correspond comme son nom l'indique à la capacité que le héro a de résisté à la force.

    Il confère au héro 2 points d’Armure contre les dégâts provoqués par la Force et ainsi qu'un bonus de +2 aux jets pour résister aux effets de ces pouvoirs. Utiliser un pouvoir bénéfique sur le personnage n'implique pas de malus, cet atout ne se déclenche qu'à la volonté du héro.

    Pour la grande résistance, le bonus d'Armure passe à 4.
\end{paperbox}

\subsubsection{Atouts de Combat}
\begin{description}[align=left]
    \item [Arme fétiche]
    	\emph{[Novice, Combat ou Tir d10+]}\\
        Le héros ne jure que par une arme qu’il connaît par c\oe{ur}.

    \item [Arme fétiche adorée]
    	\emph{[Vétéran, Arme fétiche]}\\
        Votre héros n'utilise que cette arme depuis que ces aventures en commencés, il a une histoire avec cette arme et pour rien au monde il ne s'en séparerait.

    \item [Arts martiaux]
    	\emph{[Novice, Combat d6+]}\\
        Ce personnage est entrainé aux techniques de combat à mains nues.

    \item [Maître des arts martiaux]
    	\emph{[Vétéran, Arts martiaux, Combat d10+]}\\
        Le personnage fait des dégâts de For + d6 lors d’une attaque à mains nues.

    \item [Bagarreur]
    	\emph{[Novice, Force d8+]}\\
        Votre héros a l’habitude de cogner avec ses poings, et fort !

    \item [Cogneur]
    	\emph{[Aguerri, Bagarreur]}\\
        Lorsqu’un cogneur obtient une Relance lors d’une attaque à mains nues.

    \item [Balayage]
    	\emph{[Novice, Force d8+, Combat d8+]}\\
        Cet Atout permet au personnage d’attaquer toutes les cibles adjacentes.

    \item [Grand balayage]
    	\emph{[Vétéran, Balayage]}\\
        Même chose que pour Balayage mais sans le malus.

    \item [Blocage]
    	\emph{[Aguerri, Combat d8+]}\\
        Les héros endurcis au corps à corps sont plus habiles à se défendre que les autres.

    \item [Grand blocage]
    	\emph{[Vétéran, Blocage]}\\
        Même chose que pour Blocage mais le bonus est double.

    \item [Combat à deux armes]
    	\emph{[Novice, Agilité d8+]}\\
        Votre héros n’est pas ambidextre mais sait se battre avec deux armes en même temps.

    \item [Combatif]
    	\emph{[Aguerri]}\\
        Votre héros se ressaisit vite après un coup ou une émotion forte.

    \item [Contre-attaque]
    	\emph{[Aguerri, Combat d8+]}\\
        Les combattants disposant de cet Atout sont capables de répondre instantanément aux erreurs de leurs adversaires.

    \item [Grande contre-attaque]
    	\emph{[Vétéran, Contre-attaque]}\\
        Comme contre-attaque, mais sans le malus.

    \item [Dégaine comme l’éclair]
    	\emph{[Novice, Agilité d8+]}\\
        Cet Atout permet au héros de dégainer une arme et d’attaquer dans le même round sans subir de malus.

    \item [Esquive]
    	\emph{[Aguerri, Agilité d8+]}\\
        L’expérience a appris à votre héros comment esquiver les mauvais coups.

    \item [Grande esquive]
    	\emph{[Vétéran, Esquive]}\\
        Même chose que pour Esquive en doublant les effets.

    \item [Extraction]
    	\emph{[Novice, Agilité d8+]}\\
        Votre héros utilise ses deux mains avec la même facilité.

    \item [Grande extraction]
    	\emph{[Novice, Extraction]}\\
        Comme Extraction, mais en cas de Relance sur le jet d’Agilité, aucun adversaire au contact du personnage ne bénéficiera d’une attaque gratuite.

    \item [Florentine]
    	\emph{[Novice, Agilité d8+, Combat d8+]}\\
        Combattre à la Florentine est une technique basée sur l’utilisation de deux armes à la fois.

    \item [Frappe éclair]
    	\emph{[Novice, Agilité d8+]}\\
        Une fois par round votre héros obtient une attaque de Combat gratuite contre un seul ennemi venant au contact.

    \item [Frappe foudroyante]
    	\emph{[Héroïque, Frappe éclair]}\\
        Même chose que pour Frappe éclair, mais le héros obtient une attaque de Combat supplémentaire contre chacun des ennemis venant au contact.

    \item [Frénésie]
    	\emph{[Aguerri, Combat d10+]}\\
        Frénésie en combat de mêlée permet de faire pleuvoir des coups rapides sur ses adversaires au détriment de la précision.

    \item [Frénésie suprême]
    	\emph{[Vétéran, Frénésie]}\\
        Même chose que pour Frénésie mais sans le malus.

    \item [Improvisation martiale]
    	\emph{[Aguerri, Intellect d6+]}\\
        Les héros se trouvent parfois avec du matériel non destiné au combat. Votre héros est capable d’utiliser ce genre d’objets en tant qu’armes improvisées.

    \item [Increvable]
    	\emph{[Joker, Novice, Âme d8+]}\\
        Votre héros a plus de vies qu’un camion rempli de chats.

    \item [Trompe-la-mort]
    	\emph{[Joker, Novice, Âme d8+]}\\
        Votre héros est plus dur à tuer que Raspoutine lui-même.

    \item [Instinct de tueur]
    	\emph{[Héroïque]}\\
        Votre héros déteste perdre. Dans un cas d’égalité sur un jet opposé, il gagne.

    \item [Nerfs d’acier]
    	\emph{[Joker, Novice, Vigueur d8+]}\\
        Même chose que pour Nerfs d’acier, mais 2 points de malus pour blessures sont ignorés.

    \item [Panache]
    	\emph{[Novice, \^Ame d8+]}\\
        Quand ce héros met tout son cœur dans une tache, ça se voit !

    \item [Poigne ferme]
    	\emph{[Novice, Agilité d8+]}\\
        Votre héros ignore le malus Plateforme instable pour tirer depuis un véhicule ou sur une monture en mouvement.

    \item [Rock n’ roll !]
    	\emph{[Aguerri, Tir d8+]}\\
        Les tireurs expérimentés ont appris à maîtriser le recul provoqué par les armes automatiques.

    \item [Sans pitié]
    	\emph{[Aguerri]}\\
        Le personnage peut utiliser un Jeton pour relancer un jet de dégâts, y compris pour une attaque de zone.

    \item [Tête froide]
    	\emph{[Aguerri, Intellect d8+]}\\
        Celui qui garde son calme quand les autres courent aux abris est un adversaire redoutable.

    \item [Sang-froid]
    	\emph{[Aguerri, Tête froide]}\\
        Même chose que pour Tête froide en doublant les effets.

    \item [Tireur d’élite]
    	\emph{[Aguerri]}\\
        Le héros vise juste et bien.

    \item [Tueur de géant]
    	\emph{[Vétéran]}\\
        En général plus c’est grand plus c’est dur à vaincre. Mais votre héros sait trouver les points faibles des créatures de grande taille.

\newpage

    \item [Maître Jedi]
    	\emph{[Vétéran, \^Ame d12+, Padawan]}\\
        Après sa formation et après avoir prété serment à l'ordre, votre héro pourra devenir un Maître Jedi.

    \item [Seigneur Sith]
    	\emph{[Vétéran, \^Ame d12+, Apprenti Sith]}\\
        Votre héro, après avoir prété serment à son Maître, devient à son tour un seigneur Sith.

    \item [Padawan]
    	\emph{[Aguerri, \^Ame d10+, Arcane (Force)]}\\
        La sensibilité de votre héro à la force a été remarqué et l'ordre vous accepte comme padawan sous la tutelle d'un Maître.

    \item [Apprenti Sith]
    	\emph{[Aguerri, \^Ame d10+, Arcane (Force)]}\\
        La haine et la rancoeur de votre héro à donné des idées à un Seigneur Sith qui decide de vous prendre comme apprenti.


\end{description}

\subsubsection{Atouts de commandement}

\begin{description}[align=left]
    \item [Ambidextre]
    	\emph{[Novice, Agilité d8+]}\\
        Votre héros utilise ses deux mains avec la même facilité.

    \item [Ambidextre]
    	\emph{[Novice, Agilité d8+]}\\
        Votre héros utilise ses deux mains avec la même facilité.

    \item [Ambidextre]
    	\emph{[Novice, Agilité d8+]}\\
        Votre héros utilise ses deux mains avec la même facilité.

    \item [Ambidextre]
    	\emph{[Novice, Agilité d8+]}\\
        Votre héros utilise ses deux mains avec la même facilité.

    \item [Ambidextre]
    	\emph{[Novice, Agilité d8+]}\\
        Votre héros utilise ses deux mains avec la même facilité.

    \item [Ambidextre]
    	\emph{[Novice, Agilité d8+]}\\
        Votre héros utilise ses deux mains avec la même facilité.

    \item [Ambidextre]
    	\emph{[Novice, Agilité d8+]}\\
        Votre héros utilise ses deux mains avec la même facilité.

    \item [Ambidextre]
    	\emph{[Novice, Agilité d8+]}\\
        Votre héros utilise ses deux mains avec la même facilité.

    \item [Ambidextre]
    	\emph{[Novice, Agilité d8+]}\\
        Votre héros utilise ses deux mains avec la même facilité.


\end{description}