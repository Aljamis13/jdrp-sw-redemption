\onecolumn
\section{Préface}

\lettrine{P}{ourquoi} encore un autre JdR {\jedifont Star Wars} dérivé de \citetitle{savage-worlds}. C'est vrai il en existe déjà plusieurs comme \citetitle{starwars-reloaded} ou \citetitle{starwars-unchained}. Toutes les versions que j'ai trouvé présentent des défauts: fautes d'orthographe ou de typo, traductions relative, trop ou pas assez de détails, \ldots Mais surtout, aucune des versions que j'ai trouvé n'est colaborative, il n'ai pas possible de proposer des corrections ou des améliorations ni de se forker sa propre version.

C'est pour cela que j'ai eu envie de faire une nouvelle version, une version disponible sur Githhub que n'importe qui pourra améliorer, corrigé ou forker à volonté. C'est comme ça qu'est né {\jedifont \doctitle}. Un MJ y trouvera tout ce que ses joueurs ont besoin de savoir pour créer des fiches de perso et démarrer une partie. Biensûr je n'ai pas retranscrit tout le bouquin original de Savage Worlds, mais les règles de base sont renseigné dans ce manuel afin que les joueurs n'ai pas a lire tout le bouquin de base. Le MJ lui n'y échapera pas.

De même, j'ai choisi de ne pas trop approfondir la description de l'univers de Star Wars et de me concentrer sur les adaptations de \citetitle{savage-worlds}. L'univers est largement assez connu et détaillé sur les sites spécialisés sur internet (\citetitle{website:starwars-holonet}, \citetitle{website:starwars-universe}, \ldots) et chaque MJ prendra les libertés qu'il souhaite dans son background. Ce manuel est à considérer comme le manuel du joueur, le MJ aura besoin du livre de base \citetitle{savage-worlds} pour mener la partie. Les informations présentes dans ce manuel ne suffisent pas.

\subsection{Contribution}
Comme je le disais plus haut, l'idée principale de ce document est de permettre à tous ceux qui le souhaitent de participer. Le document est donc écrit en \LaTeX et disponible sur GitHub.\\
\cite{website:jdrp-starwars-reloaded}.

\twocolumn